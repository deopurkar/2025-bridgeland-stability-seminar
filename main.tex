\documentclass{amsart}

\title{Bridgeland Stability Seminar}
\author{ANU 2025}

\newcommand{\me}[1]{\marginpar{\small{#1}}}

\begin{document}

\maketitle

\section{Why is a derived category?}
\me{Anand}

This is an important question to keep in mind.  I hope you find your own answers to this question.

The most common answer I have heard is that it is the ``right framework'' to work with complexes and their cohomology, that is, to do homological algebra.  I have also heard the counterpoint that we should just work with complexes and do honest homological algebra instead of inventing a categorical framework.  I am sympathetic to both sentiments.

\section{What is a derived category?}

Chapter 2 of \cite{fmtag} (mostly 2.1).

Questions to think about.
\begin{enumerate}
\item Exercise 2.6. Read what is expected of ``distinguished triangles'' (axioms TR1 to TR4 Chapter 1) and explain Exercise 2.6.
\item Exercise 2.25.  Describe \(D(\operatorname{vec} (k))\).

\item Exercise 2.27 is wrong.  A short-exact sequence in \(\mathcal{A}\) does not necessarily become a distinguished triangle in \(K(\mathcal{A})\) (the given procedure only works in \(D(\mathcal{A})\)).  Explain this.

  Precisely, give an example of a short exact sequence of \(\mathbf{Z}\)-modules
  \[ 0 \to A \to B \to C \to 0\]
  in which \(C\) is not isomorphic to the cone of \(A \to B\) in \(K(\mathbf{Z})\).
  Is \(C\) isomorphic to the cone in \(D(\mathbf{Z})\)?

\item Suppose \(A\) is a complex such that \(H^n(A) = 0\) for all \(n\).
  Is it true that \(A = 0\)?
  Is the same true in \(K(\mathcal{A})\)?

\item Suppose \(f \colon A \to B\) is a  map of complexes that induces the \(0\) map \(H^n(A) \to H^n(B)\) for all \(n\).
  Show, with an example, that this does not mean that \(f\) is the zero map in \(D(\mathcal{A})\).

\item The cohomology filtration.
  Explain the truncations of a complex.
  If the complex is bounded (or has bounded cohomology), explain how this leads to a ``filtration'' of a complex by its cohomology objects.

\item Prove that in \(D^b(\mathbf{Z})\), a complex is isomorphic to the direct sum of the shifts of its cohomology objects.

  This statement seems to be in tension with the (false) statement that if we have a short exact sequence
\[ 0 \to A \to B \to C \to 0\]
in \(\mathbf{Z}-\operatorname{mod}\), then \(B\) is the direct sum of \(A\) and \(C\).
Reconcile the two.
  
\end{enumerate}



\begin{thebibliography}{1}
  \bibitem[FMTAG]{fmtag} D. Huybrechts, \emph{Fourier--Mukai Transforms in Algebraic Geometry}.
\end{thebibliography}

\end{document}
